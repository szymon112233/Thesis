\newpage
\begin{center}
\large \bf
Implementacja gry ,,Turbo Slam Dunk Unleashed'' z trybem online multiplayer w środowisku Unity
\end{center}

\section*{Streszczenie}
W ramach pracy dyplomowej zaimplementowano grę komputerową Turbo Slam Dunk Unleashed w środowisku Unity. Projekt został wykonany w języku C\#. Gra została pierwotnie stworzona w środowisku Game Maker Studio, w 2016 roku. Gra została napisana od początku w nowej technologi w celu polepszenia jej jakości oraz zwiększenia możliwości dodawania nowych funkcjonalności. Ponadto została ona rozszerzona o moduł multiplayer, a także wstępnie przygotowana do przeniesienia na nowe platformy (Linux, konsole do gier).

Gotowa gra pozwala na rozgrywanie meczów koszykówki 1 vs. 1 zarówno na jednym komputerze jak i na dwóch maszynach połączonych przez internet. Mecze online można rozgrywać w trybach: ze znajomym oraz z losowym przeciwnikiem. Rozgrywka opiera się na symulacji fizyki piłki koszykowej w dwóch wymiarach. W grze dostępne jest 8 wariantów kolorystycznych piłki, 5 wariantów kolorystycznych postaci graczy oraz 2 boiska. W każdym z trybów możliwy jest wybór wspomnianych wariantów oraz długości meczu przed jego rozpoczęciem.

W ramach pracy skupiono się nie tylko na problemach programistycznych związanych z grą dwuwymiarową, ale także zaprojektowaniu rozgrywki przyjemniej w odbierze dla potencjalnych graczy.

\bigskip
{\noindent\bf Słowa kluczowe:} Unity, C\#, gra komputerowa, grafika komputerowa, Game Maker Studio

\vskip 2cm


\cleardoublepage
\newpage
\begin{center}
\large \bf
Implementation of the game ``Turbo Slam Dunk Unleashed'' in Unity platform, including online multiplayer mode 
\end{center}

\section*{Abstract}
As a part of the diploma thesis, the game Turbo Slam Dunk Unleashed has been implemented in Unity enviroment The proejct has been made using the C\# language. The Game was orginally created in Game Maker Studio enviroment in 2016. The game has been created form scratch in the new technology in order to improve the quality and to extend the abilty to add new features. Moreover, the game has been extended by online multiplayer module, and pre-prepared to be deployed to new platforms (Linux, game consoles)

The finished game allowes to play basketball matches, 1vs. 1 on a single computer as well as on two machines connecte via internet. Online matches can be played in two modes: with firend or with random opponent. The gameplay consists of a physical simulation of a two-dimmensional basketball.
There are 8 ball color variants, 5 player color variants and 2 courts available in game. In each of these modes, there's a choice beetween them as well as the choice of match length before start of it.

The focus of the thesis isn't only on the programming problems associated with two-dimmensional gameplay, but also on designing the gameplay that is considered fun to play by potential players.


\bigskip
{\noindent\bf Keywords:} Unity, C\#, computer game, computer graphics, Game Maker Studio

\vfill