\documentclass[a4paper,12pt,twoside,openany]{report}
%
% Wzorzec pracy dyplomowej
% J. Starzynski (jstar@iem.pw.edu.pl) na podstawie pracy dyplomowej
% mgr. inż. Błażeja Wincenciaka
% Wersja 0.1 - 8 października 2016
%
\usepackage{polski}
\usepackage{helvet}
\usepackage[T1]{fontenc}
\usepackage{anyfontsize}
\usepackage[utf8]{inputenc}
\usepackage[pdftex]{graphicx}
\usepackage{tabularx}
\usepackage{array}
\usepackage[polish]{babel}
\usepackage{subfigure}
\usepackage{amsfonts}
\usepackage{verbatim}
\usepackage{indentfirst}
\usepackage[pdftex]{hyperref}
\usepackage{float}
\usepackage{listings}
\usepackage{color}


% rozmaite polecenia pomocnicze
% gdzie rysunki?
\newcommand{\ImgPath}{.}

% oznaczenie rzeczy do zrobienia/poprawienia
\newcommand{\TODO}{\textbf{TODO}}


% wyroznienie slow kluczowych
\newcommand{\tech}{\texttt}

% na oprawe (1.0cm - 0.7cm)*2 = 0.6cm
% na oprawe (1.1cm - 0.7cm)*2 = 0.8cm
%  oddsidemargin lewy margines na nieparzystych stronach
% evensidemargin lewy margines na parzystych stronach
\def\oprawa{1.05cm}
\addtolength{\oddsidemargin}{\oprawa}
\addtolength{\evensidemargin}{-\oprawa}

% table span multirows
\usepackage{multirow}
\usepackage{enumitem}	% enumitem.pdf
\setlist{listparindent=\parindent, parsep=\parskip} % potrzebuje enumitem

%%%%%%%%%%%%%%% Dodatkowe Pakiety %%%%%%%%%%%%%%%%%
\usepackage{prmag2017}   % definiuje komendy opieku,nrindeksu, rodzaj pracy, ...


%%%%%%%%%%%%%%% Strona Tytułowa %%%%%%%%%%%%%%%%%
% To trzeba wypelnic swoimi danymi
\title{Implementacja gry “Turbo Slam Dunk Unleashed” z trybem online multiplayer w środowisku Unity}

% autor
\author{Szymon Piórkowski}
\nrindeksu{279065}

\opiekun{dr hab. inż. Paweł Piotrowski}
%\konsultant{prof. Dzielny Konsultant}  % opcjonalnie
\terminwykonania{28 stycznia 2017} % data na oświadczeniu o samodzielności
\rok{2017}


% Podziekowanie - opcjonalne
\podziekowania{\noindent
{\Large Podziękowania}
\bigskip

Podziękowania...

\bigskip

{\raggedleft
Szymon Piórkowski

}

}

% To sa domyslne wartosci
% - mozna je zmienic, jesli praca jest pisana gdzie indziej niz w ZETiIS
% - mozna je wyrzucic jesli praca jest pisana w ZETiIS
%\miasto{Warszawa}
%\uczelnia{POLITECHNIKA WARSZAWSKA}
%\wydzial{WYDZIAŁ ELEKTRYCZNY}
\instytut{INSTYTUT ELEKTROENERGETYKI}
\zaklad{ZAKŁAD SIECI I SYSTEMÓW ELEKTROENERGETYCZNYCH}
%\kierunekstudiow{INFORMATYKA}

% domyslnie praca jest inzynierska, ale po odkomentowaniu ponizszej linii zrobi sie magisterska
%\pracamagisterska
%%% koniec od P.W

\opinie{%
  \input{opiniaopiekuna.tex}
  \newpage
  \input{recenzja.tex}
}

\streszczenia{
  \newpage
\begin{center}
\large \bf
TYTUŁ PRACY DYPLOMOWEJ
\end{center}

\section*{Streszczenie}
Praca składa się z krótkiego wstępu jasno i
wyczerpująco opisującego oraz uzasadniającego cel pracy, trzech rozdziałów (2-4)
zawierających opis istniejących podobnych
rozwiązań, komponentów rozpatrywanychjako kandydaci do
tworzonego systemu i wreszcie zagadnień wydajności wirtualnych
rozwiązań. Piąty rozdział to opis  środowiska obejmujący opis konfiguracji
środowiska oraz przykładowe ćwiczenia laboratoryjne. Ostatni
rozdział pracy to opis możliwości dalszego
rozwoju projektu. 

\bigskip
{\noindent\bf Słowa kluczowe:} praca dyplomowa, LaTeX, jakość

\vskip 2cm


\begin{center}
\large \bf
THESIS TITLE
\end{center}

\section*{Abstract}
This thesis presents a novel way of using a novel algorithm to solve complex
problems of filter design. In the first chapter the fundamentals of filter design
are presented. The second chapter describes an original algorithm invented by the
authors. Is is based on evolution strategy, but uses an original method of filter
description similar to artificial neural network. In the third chapter the implementation
of the algorithm in C programming language is presented. The fifth chapter contains results
of tests which prove high efficiency and enormous accuracy of the program. Finally some
posibilities of further development of the invented algoriths are proposed.

\bigskip
{\noindent\bf Keywords:} thesis, LaTeX, quality

\vfill
}

\begin{document}
\maketitle

%-----------------
% Wstęp
%-----------------
\chapter{Wstęp}
Celem niniejszej pracy jest przeniesienie gry Turbo Slam Dunk Unleashed do środowiska Unity oraz dodanie do niej modułu online multiplayer. Ponadto, podjęte zostały kroki w celu ułatwienia przygotowania projektu do uruchomienia na nowych platformach. W tym celu gra musiała zostać napisana od zera w języku C\#. Pierwotnie gra została stworzona przez autora niniejszej pracy w środowisku Game Maker.

\section{Gry komputerowe}
Początków gier komputerowych można szukać jeszcze przed pojawieniem się pierwszych komputerów osobistych z interfejsem graficznym. Mało osób wie, że pierwsze gry były produkcjami akademickimi. Takie gry technicznie nie różnią się bardzo od innych programów komputerowych. Są to zwykle programy realizujące zdania takie jak renderowanie, obliczanie logiki gry czy przetwarzanie danych wejściowych od użytkownika w czasie rzeczywistym. Wykorzystywana jest do tego tzw. pętla główna która jest wykonywana określoną ilość razy na sekundę. Głównym celem gier jest dostarczenie użytkownikowi rozrywki.

W dzisiejszych czasach gry są już niemal wszędzie. Mimo tak krótkiej historii istnieją już klasyki, które zna niemalże każdy gracz. Zdecydowana większość osób młodych miała z już nimi styczność. Popularność gier można zawdzięczać chociażby faktowi, że gry powstają na niemalże każdą platformę elektroniczną - od starszych telefonów komórkowych z dedykowanymi systemami operacyjnymi, po nowoczesne konsole do gier. Do ciekawszych przykładów platform na które stworzone zostały gry można zaliczyć: oscyloskopy, czytniki ebooków, zegarki czy telewizory typu SmartTV. Zjawisko to może być porównane do innych dziedzin kultury.

Sam problem stworzenia gry komputerowej może wydawać się pozornie prosty, ze strony jej odbiorcy, jednak są to jedne z najbardziej zaawansowanych dzieł ludzkich. Poza kompleksowymi problemami natury programistycznej czy algorytmicznej, dobra gra musi przyciągać uwagę użytkownika i zaspokajać jego ciekawość oraz potrzeby rozrywki. Do osiągnięcia tych celów nie rzadko korzysta się z bardzo wielu dziedzin nauki, między innymi: psychologia, fizyka lub biologia.

Ponadto branża gier komputerowych nie różni się bardzo od branży filmowej czy muzycznej. Powoli zaczyna doganiać ich rozmiar w niektórych krajach, a globalny rynek gier komputerowych przewyższył swoją wartością przemysł filmowy już w 2013 roku. W Polsce z roku na rok branża growa rośnie, a polskie gry są jednym z lepszych towarów eksportowych\cite{canalplus_gry}. Zdecydowanie mają przed sobą ciekawą przyszłość. Dlatego też zagadnienie to jest głównym tematem tej pracy. 

\section{Motywacja}
Pierwotnie Gra Turbo Slam Dunk Unleashed została wykonana w środowisku Game Maker Studio. Dawało to wystarczające możliwości jak na prostotę gry, jednak narzucało również dużo ograniczeń oraz komplikacji. Game Maker jest prostym narzędziem do tworzenia gier dwuwymiarowych. Struktura projektu, którą narzuca to środowisko, sprawia że stosunkowo trudno rozwijać w nim projekty o większej skali. Game Maker korzysta z języka skryptowego Game Maker Language (GML), który składnią przypomina C i Javascript. Skrypty są zwykle krótkimi plikami realizującymi proste zadania. Przez to, naturalnie wraz z wzrostem projektu robi się ich duża ilość, a wyszukiwanie konkretnych zmian staje się coraz trudniejsze. Z tych powodów pisanie warstwy sieciowej wprowadziłoby chaos w wystarczająco skomplikowanym już projekcie.  Ponadto wsparcie dla konsol, mimo swojego istnienia, jest znacznie ograniczone w stosunku do tego w Unity.

Środowisko Unity pozwala pokonać wszystkie te problemy, dodając przy tym projektowi nowego potencjału dla rozwiązywania niektórych, zarówno algorytmicznych jak i designowych problemów w grze. Wsparcie samych twórców silnika jest stabilniejsze w przypadku Unity z prostego powodu - komercyjnych gier tworzonych za jego pomocą jest znacznie więcej. Najważniejszym powodem przepisania gry nie była sama technologia, co postęp doświadczenia twórcy. Finalna wersja gry powstała w 2016 roku. Po 3 latach umiejętności autora znacznie się poprawiły i projekt może zdecydowanie na tym skorzystać. 

\section{Założenia}
Gry komputerowe nie mają jasno określonej granicy skończenia. Zawsze można coś poprawić, dodać czy zmienić. W takim projekcie ważne jest żeby wyznaczyć jasne cele i wymagania które doprowadzą do stworzenia skończonego produktu. Dlatego też zostały stworzone założenia, które musi spełniać gra żeby uznać ją za kompletną:

\begin{itemize}
    \item Przeniesienie gry w wersji podstawowej do środowiska Unity
    \begin{itemize}
        \item Mecz 2 graczy na 1 komputerze,
        \item Rzucanie i zabieranie piłki, faule,
        \item Liczenie punktów, stan meczu,
        \item Wybór postaci, piłki, planszy oraz czasu meczu.
    \end{itemize}
    
    \item Dodanie trybu online multiplayer
    \begin{itemize}
        \item Mecz 2 graczy na 2 komputerach przez sieć,
        \item Synchronizacja pozycji graczy oraz piłki,
        \item Synchronizacja stanu meczu - punkty, faule, czas i inne,
        \item Synchronizacja wyboru ustawień meczu - wygląd graczy, piłki, mapa oraz czas meczu,
        \item Możliwość dołączenia do losowego gracza lub gry ze znajomym,
        \item obsługa błędów związana z problemami z połączeniem.
    \end{itemize}
    
    \item Przygotowanie gry do przeniesienia na nowe platformy
    \begin{itemize}
        \item Przygotowanie warstwy abstrakcji wejść,
        \item Optymalizacja logiki i warstwy sieciowej.
    \end{itemize}
\end{itemize}



%-----------------
% Opis Środowiska
%-----------------
\chapter{Opis Środowiska}
W tym rozdziale znajduje się opis narzędzi przy pomocy których został zrealizowany projekt gry.
\section{Unity}
Unity jest częściowo darmowym silnikiem do tworzenia gier stworzonym w 2005 roku. Od tego czasu silnik był wielokrotnie modyfikowany i zostało opublikowane wiele nowych wersji. Obecnie najnowsza stabilna wersja oznaczona jest numerem 2018.3\cite{unity_wiki}. Silnikiem do gry nazywamy zestaw narzędzi, oraz sprawdzonych algorytmów które często używane są do tworzenia gier. Pierwsza wersja Unity została stworzona przez 3 kolegów - Davida Helgason, Joachima Ante oraz Nicholasa Francisa w Dani. Unity posiada kilka typów licencji, podstawowa jest darmowa jednak po przekroczeniu określonego przez Unity półapu przychodów, należy kupić licencję wyższego poziomu. Płatne licencje nie dają tylko takiej możliwości, rozszerzają one nieco możliwości edytora a także zwiększają wsparcie developerów. W przypadku najdroższej wersji - Enterprise, użytkownicy silnika mają nawet dostęp do kodu źródłowego silnika.

\begin{table}[h!]
\centering
\begin{tabular}{c|cccc}
Nazwa Wersji & Limit przychodów & Wsparcie premium & Cena \\ \hline
Personal & \$100,000 & NIE & Darmowa \\
Plus & \$200,000 & NIE & \$420 rocznie \\
Pro & Nielimitowane & TAK & \$1500 rocznie \\
Enterprise & Nielimitowane & TAK & Indywidualnie \\
\end{tabular}
\caption{Zestawienie licencji Unity |SOURE DODAC|}
\label{table_unity_versions}
\end{table}

\begin{figure}[!htbp]
	\begin{center}
\centering
\includegraphics[scale=0.2]{\ImgPath/rys/unity_home.png}
\end{center}
	\caption{Domyślny widok edytora Unity w projekcie gry}
	\label{unity_home}
\end{figure}

Na chwilę obecną silnik obsługuje 27 platform: iOS, Android, Tizen, Windows, Universal Windows Platform, macOS, Linux, WebGL, PlayStation 4, PlayStation Vita, Xbox One, Wii U, 3DS, Oculus Rift, Google Cardboard, SteamVR, PlayStation VR, Gear VR, Windows Mixed Reality, Daydream, Android TV, Samsung Smart TV, tvOS, Nintendo Switch, Fire OS, Facebook Gameroom, Apple's ARKit, Google's ARCore, and Vuforia. Jest to dość długa lista jak na silnik do tworzenia gier. Zwykle ograniczają się one do mniejszej liczby platform. Dlatego też gra stworzona w tym silniku ma bardzo duży potencjał do zaistnienia na większej ilości platform a co za tym idzie - zyskanie nowych graczy. Samo przenisienie między platformą jest reklamowane przez Unity jako bardzo prosta akcja - wystarczy jeden przycisk. W praktyce jednak zależy to bardzo od typu gry oraz tego jak wygląda architektura samej gry.

Sam silnik napisany jest w językach C, C++ oraz C\#. W pierwszych dwóch napisane są podstawy silnika takie jak renderowanie, logika czy obsługa wejść. Trzeci z tych języków został użyty do napisania warstwy wizualnej - GUI edytora oraz do stworzenia podstawowych komponentów. Może on być równiwerz używany przez twórców gier do tworzenia własnych komponentów lub rozszerzeń silnika. Unity przyjmuje podejście komponentowe. Typ obiektu w grze definiuje zestaw jego komponentów. Komponenty są mniejszymi elementami które realizują określone zdania. Do podstawowych typów komponentów można zaliczyć miedzy innymi:
\begin{itemize}
    \item \textit{Transform} - Posiada go każdy GameObject, służy do określenia pozycji, rotacji oraz skali obiektu w przestrzeni trójwymiarowej sceny. Transformy tworzą hierarchie sceny podobną do drzewa. Dzięki temu możemy np. tworzyć skompilowane obiekty na scenie a następnie poruszać je całe poprzez transalcje jednego, nadrzędnego transforma.
    \item \textit{Camera} - Podstawa renderingu, wyświetla na danym ekranie część sceny na którą aktualnie obejmuje. Posiada 2 tryby projekcji - ortogonalną oraz perspektywiczną. Z zasady pierwszy typ wykorzystywany jest dla gier dwuwymiarowych natomiast drugi dla trójwymiarowych.
    \item \textit{Sprite Renderer} - Komponent odpowiedzialny za rysowanie dwuwymiarowych obrazków - tzw. Spritów. Posiada różne tryby rysowania, obroty w osiach X i Y, kolejność sortowania oraz możliwość przemnożenia kolorów obrazka przez jeden wybrany kolor,
    \item \textit{Box Collider 2D} - Komponent ten umożliwia rejestrowanie kolizji - fundamentalnej funkcjonalności gry. Kolizje występują kiedy dwa obiekty są w kontakcie, a także dostarczają szczegółowych informacji o tych kontaktach. Typ Box jest prostszym kształtem, istnieją też inne bardziej lub mniej skomplikowane kształty colliderów.
    \item \textit{Rigidbody 2D} - Podstawa symulacji fizyki dwuwymiarowej w Unity. Obiekt posiadający ten komponent musi również posiadać komponent typu Collider 2D żeby móc odbierać informacje o kolizjach. Posiada zestaw zmiennych umożliwiających symulację fizyki: masa, pozycja, prędkość, prędkość kątowa oraz środek ciężkości. Ulega też grawitacji jeżeli taka jest zdefiniowana na scenie.
    \item \textit{Animator} - Odpowiada za animacje niemalże dowolnej wartości obiektu. Poprzez animacje możemy rozumieć zmianę wartości w czasie. Najczęściej służy do zmiany pozycji obiektów, ale można nim także zmieniać pojedyncze zmienne komponentów takie jak kolor czy sprite. Animator jest realizowany w postaci maszyny stanów.
    \item \textit{Canvas} - Podstawa interfejsu użytkownika. Definiuje przestrzeń w której wyświetlane są jego elementy. Może być ona wyświetlana jako narzuta na okno gry, narzuta na konkretną kamerę, lub płaszczyzna w przestrzeni trójwymiarowej.
    \item \textit{Mono Behavior} - Komponenty tego typu są skryptami w języku C\#. Można więc swobodnie je definiować i stworzyć w nich niemalże dowolną funkcjonalność. Więcej informacji o skryptowaniu znajduje się w sekcji "Język C\# oraz Mono"
\end{itemize}
Powyższe elementy są najczęściej wykorzystywane podczas tworzenia gry dwuwymiarowej. Istnieje szereg komponentów o funkcjonalności trójwymiarowej, ale jako że gra opisywana w tej pracy operuje tylko na dwóch wymiarach zostały one pominięte.

\section{Język C\# oraz Mono}

Do programowania rozgrywki oraz logiki gry używa się języka C\#. Nie jest on jednak natywnym językiem silnika. Żeby możliwe było obsługiwanie Skryptów w tym języku Unity używa zewnętrznej platformy zwaną Mono. Mono jest open-sourcowym projektem opartym na frameworku .Net\cite{about_mono}. Pozwala między innymi na łatwe budowanie aplikacji na wiele platform.

\begin{figure}[!htbp]
	\begin{center}
\centering
\includegraphics[scale=0.6]{\ImgPath/rys/mono_architecture_scripting.png}
\end{center}
	\caption{Architektura implementacji C\# jako język skryptowy w Mono}
	\label{mono_architecture_scripting}
\end{figure}

W przeszłości do podobnych celów wykorzystywało się języki skryptowe. Wiele silników posiadało własne mniej lub bardziej wydajne implementacje. Często jednak wielkość skryptów przekraczała możliwości interpretatora lub nawet silnika skryptowania przez co możliwości były bardzo ograniczone. Mono używa różnych języków do różnych celów, a więc traktuje je jako narzędzia które służą do rozwiązywania innych problemów.

Podczas gdy ważna, krytyczna część aplikacji może być napisana w wydajnym nisko poziomowym języku jak C, częściami takimi jak interfejs użytkownika czy interakcje z użytkownikiem zajmuje się język wyższego poziomu który nie jest tak samo wydajny ale pozwala zmniejszyć ilość wymaganych linijek kodu do zrealizowania konkretnego zadania. Kod napisany w takim języku jest zdecydowanie wolniejszy od kodu natywnego ale w porównaniu z popularnymi językami skryptowymi takimi jak np. LUA jest zdecydowanie szybszy. Mono umożliwia bardzo proste wołanie metod w języku natywnym przez co łatwo jest podzielić zadania na te które powinny być napisane optymalnie, a te które nie wymagają aż takiej oszczędności. Wspierane jest wiele języków wyższego poziomu do których należą miedzy innymi: C\#, Java, F\#, Python czy JavaScript.

Język C\# jest stworzonym dla Microsoft obiektowym językiem programowania. Kod napisany w tym języku kompilowany jest do Common Intermediate Language (CIL) który jest z kolei wykonywany w środowisku uruchomieniowym  takim jak .Net czy właśnie Mono.  


\begin{figure}[!htbp]
	\begin{center}
\centering
\includegraphics[scale=0.43]{\ImgPath/rys/CIL.png}
\end{center}
	\caption{Schemat pokazujący proces kompilacji języka wykorzystującego CIL}
	\label{CIL}
\end{figure}

Język ten jest silnie typowany, deklaratywny, imperatywny oraz funkcyjny. Powstał w około 2000 roku i został zaakceptowany jako standard przez ECMA (ECMA-334) oraz ISO (ISO/IEC 23270:2006). Posiada wiele przydatnych funkcji:
\begin{itemize}
    \item Odśmiecanie pamięci - C\# automatyczne zarządzą pamięcią poprzez liczenie referencji do obiektów. Jeżeli do danego obiektu nie prowadzi żadna referencja jest on niszczony podczas przebiegu tzw Garbage collectora
    \item Delegaty oraz Zdarzenia - bardzo wygodny sposób na kontrolowanie wykonywania kodu i uruchomienie go podczas zdarzeń. Jest to pewnym rozszerzeniem wskaźników które możemy spotkać w językach niższego poziomu.
    \item Refleksja i atrybuty klas - Podczas działania programu istnieje możliwość analizy jego struktury z poziomu tego kodu. Przydaje się chociażby podczas debugowania kodu czy innych zastosowaniach które korzystają z nieznanej podczas kompilacji struktury kodu.
    \item Typy generyczne - mechanizm zbliżony do działania szablonów w C++. Pozwala na tworzenie całych klas operujących na danych o nieznanym typie ale posiadającym konkretny zestaw funkcji.
\end{itemize}

\section{Dodatkowe narzędzia}
Do stworzenia kompletnej gry teoretycznie wystarczy sam silnik jednak używnie zewnętrznych programów do tworzenia plików znacznie poprawia jakość zarówno samej produkcji jak i pracy nad nią. Poniżej znajduje się lista użytego oprogramowania:

\begin{itemize}
    \item Microsoft Visual Studio 2017 - środowisko programistyczne wspierające wiele języków. Możliwe jest doinstalowanie pakietu integracji z Unity. Poza edycją skryptów pozwala na debugowanie projektu poprzez wbudowany debugger.
    \item Git - Darmowy i open sourcowy program służący do kontroli wersji.Jest to program konsolowy. Bardzo przydaje się do śledzenia zmian, a także do łatwego tworzenia kopi zapasowej plików projektowych.
    \item Git Extensions - Darmowy i open sourcowy program do obsługi gita. Uzupełnia gita o graficzny interfejs co ułatwia pracę przy dużych projektach. Zamyka też wiele skomplikowanych komend w przystępne guziki co uprzyjemnia pracę z gitem.
    \item Paint.Net - Darmowy i open sourcowy program do edycji grafiki. Napisany w języku C\# z wykorzystaniem frameworku .Net. Bardzo prosty w obsłudze program dzięki któremu poprawki grafiki zostały wykonane szybko i precyzyjnie bez utraty jakości.
    \item TexturePacker - Program do pakowania obrazków w tzw. atlasy. Taki sposób przechowywania obrazków umożliwia prostsze animowanie, jest wydajniejszy pamięciowo oraz pozwala ominąć kilka innych problemów związanych z importowaniem dużej ilości obrazków do projektu z grą.
\end{itemize}

%-----------------
% Opis gry “Turbo Slam Dunk Unleashed”
%-----------------
\chapter{Opis gry “Turbo Slam Dunk Unleashed”}

\section{Historia}

Proces tworzenia gry rozpoczął się w 2015 roku, a jej pełna wersja została udostępniona w serwisie Gamejolt.com 3 czerwca 2016 roku\cite{gamejolt_page}. Na początku był to projekt hobbystyczny, który potem zamienił się w projekt kołowy i był tworzony w ramach Koła Naukowego Twórców Gier "Polygon". Finalna wersja została zaprezentowana na jednym z pokazów projektów kołowych.
\begin{figure}[!htbp]
	\begin{center}
\centering
\includegraphics[scale=0.25]{\ImgPath/rys/gamejolt_page.png}
\end{center}
	\caption{Strona główna gry w serwise Gamejolt.com}
	\label{gamejolt_page}
\end{figure}

\section{Ogólny opis gry}
Głównym elementem rozgrywki są mecze koszykówki miedzy dwoma graczami rywalizującymi na jednym komputerze. Gracze używają jednej klawiatury lub padów żeby poruszać swoimi postaciami w świecie gry. Istnieje jedna piłka, której wrzucenie do kosza przeciwnika powoduje zdobycie punktu. W grze istnieją również faule. Faul to sytuacja gdy jeden gracz uderzy drugiego żeby zabrać mu piłkę  w które gracz wyskoczył z piłką i jej nie rzucił - tak jak w prawdziwej koszykówce. Faule są liczone i odejmowane od zdobytych punktów pod koniec meczu. Mecz trwa określony czas, domyślenie 3 minuty i po tym czasie liczone są punkty i wyświetlany jest werdykt meczu.

\section{Opis elementów gry}

Gra jest bardzo prosta i składa się z kilku podstawowych elementów.
Poniżej znajduje się obrazek pokazujący graficzne reprezentacje wymienionych elementów.
\begin{enumerate}
    \item Plansza - Świat gry w którym poruszają się gracze, głównie są to elementy wizualne
    \item Postacie graczy - W grze istnieją 2 instancje. Jest ona sterowana przez człowiek poprzez zestaw wejść. Może chodzić w lewo lub w prawo, skakać, rzucać piłkę oraz uderzać. Dwie ostatnie akcje możliwe są zarówno podczas stania jak i w locie. Skok służy do zwiększenia zasięgu rzutu lub po prostu lepszego do niego ustawienia. Obrót gracza jest niemożliwy w powietrzu, pozwala to na dokładniejsze dostosowanie swojej pozycji do rzutu. Rzut piłki służy głownie do umiejscowienia jej w obręczy kosza. Po rozpoczęciu przyciskania klawisza odpowiedzialnego za rzut, gracz traci możliwość chodzenia i jest w trakcie rzutu. Podczas tego czasu pokazuje się wskaźnik siły rzutu który oscyluje między minimalną i maksymalna siłą rzutu. Po puszczeniu przycisku piłka jest rzucana zgodnie z siłą która była wskazana w tym momencie. Uderzanie służy do zabierania piłki drugiemu graczowi. Po przyciśnięciu klawis0za uderzenia rozpoczyna się animacja uderzenia i jeżeli trafiliśmy ręką w punkt w którym znajduje się piłka piłka jest wybijana drugiem graczowi. W trakcie trwania animacji, możliwość chodzenia jest zablokowana.
    \item Piłka - Główny element, służy do zdobywania punktów przez graczów. Jej fizyka jest symulowana w 2 wymiarach, przez co uzyskujemy bardzo ciekawe efekty jej wyrzutu. Jeżeli znajdzie się w obręczy kosza, naliczane są 2 punkty graczowi atakującemu ten kosz. Jeżeli piłka została wyrzucona z odpowiednio dużej odległości, trafienie daje dodatkowy punkt. Jeżeli piłka wyleci poza boisko to jej pozycja resetuje się do środka planszy, a gracze pojawiają się na swoich początkowych pozycjach.
    \item Kosze -  W grze istnieją 2 instancje. Trafienie do jednego z nich dodaje punkty. Ich wygląd różni się zależnie od planszy, ale sama funkcjonalność pozostaje niezmienna.
    \item Tablica Wyników - Pokazuje aktualny stan meczu - pozostały czas, ilość punktów oraz ilość fauli każdego z graczy. na środku tablicy znajduje się grafika piłki, której kolor zmiana się zależnie od wyniku meczu.
\end{enumerate}

\begin{figure}[!htbp]
	\begin{center}
\centering
\includegraphics[scale=0.6]{\ImgPath/rys/game_elements.png}
\end{center}
	\caption{Elementy gry na zrzucie ekranu z aplikacji}
	\label{game_elements}
\end{figure}

\section{Inne funkcjonalności}
Wyżej wymienione elementy stanowią podstawę gry. Teoretycznie wystarczą one do przeprowadzenia pełnej rozgrywki jednak, nie wydaje się ona zbyt ciekawa. Gra posiada wiele funkcjonalności które ułatwiają rozgrywkę, zwiększają jej komfort, bądź też jej odczucie przez gracza:
\begin{itemize}
    \item Ekran ustawień - pozwala na ustawienie klawiszy które odpowiedzialne są za konkretne akcje w grze, oraz ustawienie poziomu głośności dźwięków. Ustawienia te są zapisywane w pliku .ini przez co po restarcie gry nie trzeba ich od nowa ustawiać. 
    \item Ekran ustawień meczu - pozwala na zmianę ustawień pojedynczego meczu. Zmienione może zostać:
    \begin{itemize}
        \item Czas meczu
        \item Wygląd postaci każdego z graczy
        \item Wygląd piłki
        \item Plansza
    \end{itemize}
    \item Dźwięki - zwiększają immersję i stanowią dodatkową informacje zwrotną z gry. W grze można usłyszeć dźwięki tła z plansz, uderzeń piłki o różne powierzchnie, gwizdka sygnalizujące faul, oraz dźwięki wyboru przycisku w menu.
    \item Efekty cząsteczkowe - Małe dodatki graficzne które sprawiają że gra wygląda ciekawiej. Można do nich zaliczyć efekt sypiącego się piasku gdy postać rusza. 
    \item Efekty specjalne - Głównie lekkie trzęsienie kamerą przy uderzeniach piłki o kosz. Stanowią dodatkową informacje zwrotną z gry.
    \item Dynamiczna kamera - Kamera która zawsze pokazuje w kadrze piłkę lub gracza z piłką. Gracz z piłką nie jest na środku, ponieważ kamera pokazuje zdecydowanie więcej strony w którą jest aktualnie zwrócony, tak żeby można było swobodnie rzucać.
    \item Obsługa padów - Gracze mogą nie chcieć używać klawiatury do gry z wielu powodów, dlatego też gra w pełni wspiera obsługiwanie jej tzw. padem czyli specjalnym kontrolerem stworzonym właśnie do grania w gry. Dzięki temu każdy gracz może wybrać swój preferowany sposób obsługi i nie musi dzielić klawiatury z przeciwnikiem.
    \item Osiągnięcia - Strona Gamejolt.com posiada system osiągnięć. Każdy użytkownik może je odblokować jeżeli zostały one zaimplementowane przez developera. W grze istnieją 3 osiągnięcia. Osiągnięcia są często ciekawym dodatkiem do gry i przedłużają czas jaki można spędzić                 nad grą.
    \item Znaczniki pozycji - Ponieważ postacie graczy nie zawsze są w kadrze, gracze nie posiadają informacji o swojej pozycji. Dzięki prostym strzałkom w kolorze gracza ta informacja jest zachowana. Ponadto, ponieważ możliwa jest zmiana wyglądu kontrolowanej postaci, a nawet możliwy jest wybór takiego samego wyglądu dla obu graczy, nad ich głowami widnieją znaczniki z numerem gracza.
\end{itemize}


\begin{figure}[!htbp]
	\begin{center}
\centering
\includegraphics[scale=0.5]{\ImgPath/rys/match_setup_old.png}
\end{center}
	\caption{Ekran ustawień meczu}
	\label{match_setup_old}
\end{figure}


%-----------------
% Implementacja
%-----------------
\chapter{Implementacja nowej wersji gry}

Tak jak zostało już wspomniane wcześniej, gra została przepisana od zera w środowisku Unity. W tym rozdziale znajduje się szczegółowy opis implementacji.

\section{Zmiany w stosunku do oryginalnej wersji gry}

Oryginalnej gra była atrakcyjna dla graczy, jednak z perspektywy czasu jej design może być poprawiony w wielu aspektach. Niektóre aspekty okazały się nietrafione a inne po prostu mogą być ulepszone o nowe pomysły. W obecnej wersji zostały wprowadzone następujące zmiany w stosunku do wersji oryginalnej:
\begin{itemize}
    \item Dodanie trybu online multiplayer - W oryginalnej grze możliwa była tylko gra na jednym komputerze. Dlatego też niezbędna była obecność 2 graczy przy jednym urządzeniu. Dzięki wprowadzeniu takiego trybu, gracze nie muszą być teraz fizycznie w jednym miejscu. Nie muszą też współdzielić klawiatury.
    \item Zmiana rozmiaru okna gry - Oryginalna gra jest wyświetlana w oknie o rozdzielczości 800x600. Jest to zdecydowanie przestarzała rozdzielczość nie mówiąc już o jej współczynniku proporcji - 4:3. Użytkownicy posiadają monitory o wile większych rozdzielczościach najczęściej w 16:9. Dlatego też gra zaczęła wspierać rozdzielczość 1920x1080  nazywaną czasami FullHD jako domyślną. Większość monitorów komputerów PC operuje na takiej rozdzielczości lub innych o takim samym współczynniku proporcji. Tak długo jak jest on zachowany okno gry skaluje się poprawnie i widoczne są wszystkie elementy gry.
    \item Poprawa kamery - Kamera nie różni się bardzo od podstawowej wersji, jednak zostały wprowadzone małe usprawnienia. Śledzony jest teraz każdy gracz oraz piłka, zamiast tylko gracza z piłką lub samej piłki. Dzięki temu kluczowe elementy gry zawsze znajdują się w polu widzenia. Przed każdym graczem widoczna jest też konkretna przestrzeń, tak żeby kosze były widoczne. Dzięki takiemu rozwiązaniu możliwe jest też zobaczenie całej mapy na ekranie.  
    \item Zmiana systemu kolorowania graczy - W wersji oryginalnej warianty kolorystyczne graczy były realizowane poprzez podmianę spritów gracza. W takim rozwiązaniu dla każdego wariantu kolorystycznego musiał powstać zestaw wszystkich klatek animacji. W wersji obecnej zmiana koloru odbywa się w tzw. shaderze. Jest to część kodu wykonywana na karcie graficznej podczas rednerowania. Shader korzystając z maski zamienia kolory gracza w każdej animacji, przez co możliwe jest stworzenie niemalże nieskończonej liczby wariantów, bardzo tanim kosztem. Rozwiązanie te jest też dużo wydajniejsze pamięciowo - każdy nowy wariant nie wymaga tworzenia plików które ładowane są do pamięci.
    \item Zmiany zasad rozgrywki - W wersji oryginalnej wyrzucenie piłki poza plansze oznaczało restart pozycji graczy oraz piłki za każdym razem. W tej wersji piłka jest po prostu odrzucana ze strony z której wyleciała. Pozwala to zachować dynamikę gry i jest nieco humorystycznym akcentem w grze. Ponadto zdobycie punktów, po prostu resetuje pozycję graczy zamiast ustawiać ich w pozycji zależenie od tego kto zdobył punkt i jaki kosz został zaatakowany.
    \item Usunięcie achievementów - W oryginalnej wersji nie ciszyły się zbytnią popularnością. Były też związane bezpośrednio z serwisem Gamejolt.org na którym nowa wersja nie będzie dostępna. Możliwe jednak że zostaną one dodane w przyszłości dla innych platform.
    \item Usunięcie znaczników pozycji - Z powodu zmian w kamerze, postacie zawsze widoczne są na ekranie. Nie ma więc potrzeby stosowania znaczników, tak jak było to zrobione w poprzedniej wersji gry.  
    \item Usuniecie obsługi padów - Ponieważ system wejść został napisany od zera, praca nad nim zajęła dużo czasu. Zrobienie dobrej obsługi innych kontrolerów niż standardowa klawiatura nie jest trywialnym zadaniem zwłaszcza jeżeli chodzi o nawigowanie po interfejsie użytkownika. Zadanie to jednak musi zostać zrealizowane jeżeli gra miałaby ukazać się na konsolach do gier.
    \item Brak mniejszych szczegółów - W oryginalnej wersji istnieje dużo drobnych szczegółów, których nie widać na pierwszy rzut oka. Sprawiają one że gra wydaje się bardziej kompletna. Można do nich zaliczyć efekty cząsteczkowe, specjalne czy dźwiękowe. Praca nad takimi szczegółami jest bardzo czasochłonna, a granie jest możliwe bez nich, dlatego też ich implementacja została odłożona na późniejszy czas.
\end{itemize}

\section{Implementacja gry z wykorzystaniem środowiska Unity}
Główną sceną gry jest scena o nazwie "Main Menu". Jej głównym celem jest umożliwienie graczowi nawigacji po grze. Jest też ona pierwszą rzeczą jaką gracz zobaczy w grze dlatego ważne jest żeby była ciekawa. Po lewej stronie znajduje się panel przycisków, a nad nimi umiejscowione jest logo gry wraz z prostą animacją. W tle widać też spadające piłki koszykowe.

\begin{figure}[!htbp]
	\begin{center}
\centering
\includegraphics[scale=0.3]{\ImgPath/rys/main_menu.png}
\end{center}
	\caption{Ekran głównego menu}
	\label{main_menu}
\end{figure}

Zdecydowana większość sceny jest interfejsem użytkownika. Dlatego też znajduje nadrzędnym elementem sceny jest obiekt \textit{MainMenuUI}. Posiada on komponent Canvas oraz wszystkie wymagane przez niego komponenty czyli CanvasScaler oraz Graphics Raycaster. Posiada również Mono Behaviour "MainMenuUI"  który odpowiada za całą jego logikę którego kod znajduje się w dodatku \ref{code_MainMenuUI}. Canvas ustawiony jest w trybie narzutu na ekran. Domyślnym ekranem jest ekran pierwszy, jako że gra korzysta z tylko jednego ekranu. Kolejność sortowania w warstwie tego komponentu wynosi 0. Zostało przyjęte że jest to domyślny widok gracza i wszystko co ma być przed nim będzie mieć mniejszą wartość kolejności, natomiast wszystko co za  - większą. komponent Canvas Scaler odpowiada za skalowanie Canvasa zależenie od podanych parametrów. Tryb skalowania ustawiony jest na Scale With Screen Size czyli skalowanie zależne od dostępnej rozdzielczości ekranu. Rozdzielczość referencyjna to rozdzielczość pod którą przygotowywany jest UI. Następną w kolejność własnością jest tryb dopasowania do ekranu. Tutaj ustawiona jest wartość Match Width or Height - dopasuj do szerokości lub wysokości. Wartość dopasowania pomiędzy szerokością a wysokością ustawiona jest na 0.5. Działa to tak że Canvas skaluje równo względem szerokości jak i wysokości. Dzięki takiej konfiguracji UI wygląda dobrze na wszystkich ekranach o współczynniku proporcji 16:9 i podobnych. Komponent Graphics Raycaster odpowiada za odczytywanie pozycji myszki na danym Canvasie przez co może wysyłać eventy do konkretnych elementów UI takich jak guziki czy tekst. Skrypt MainMenuUI posiada tylko referencje do innych obiektów tak żeby nie musieć ich wyszukiwać podczas działania programu. 

\begin{figure}[H]
	\begin{center}
\centering
\includegraphics[scale=0.75]{\ImgPath/rys/GOMainMenuUI.png}
\end{center}
	\caption{Ekran inspektora po wybraniu obiektu MainMenuUI}
	\label{GOMainMenuUI}
\end{figure}

Dzieci obiektu MainMenuUI są już elementami samego UI. W ich skład wchodzą między innymi obiekty Grass, Basket oraz Logo, które są prostymi obrazkami. Posiadają jedynie komponenty typu RectTransform oraz Image. RectTransform odpowiada za pozycje obiektu na prostokątnym Canvasie. Komponent Image wyświetla teksturę na obszarze ograniczonym rozmiarem obiektu ustawionym w RectTransform. Tryb wyświetlania tekstury ustawiony jest na prosty - jest to statyczny obraz który nie zmienia się przez cały czas jaki jest wyświetlany. Kolejnym elementem jest animacja światła pod logiem. Jest to taki sam prosty obrazek jak poprzednie z tym że posiada dodatkowy komponent typu Animation. Odpowiada on za realizowanie prostych animacji. w tym przypadku jest to prosty obrót obrazka o 90 stopni w zapętleniu. Ponieważ obrazek jest symetryczny uzyskujemy efekt nieskończonego obrotu obrazka.

Dwa ostatnie dzieci to panele zawierające kolejne elementy UI. Panele pozwalają na łatwiejszą organizacje interfejsu użytkownika. Dzięki nim prościej jest też zrealizować np proste okno. Technicznie są to po prostu puste obiekty zawierające odpowiednio skonfigurowany komponent RectTransform oraz komponent typu Image. Pierwszy prostszy panel nazwany "CreditsPanel" jest oknem w którym znajdują się informacje o twórcach gry. Domyślnie jest on nieaktywny. zawiera obiekt który realizuje tekst poprzez komponent typu Text. Komponent ten zawiera pola takie jak: Text:, Font, Font Size, Alignment czy Color. W polu Text wpisywany jest tekst który jest wyświetlany na ekranie. To właśnie tutaj wpisane zostały informacje o twórcach. Font Size odpowiada za rozmiar czcionki. Tutaj wartość wynosi 63 ale zależna jest ona od tego ile tekstu jest wyświetlane - za duży rozmiar czcionki sprawi że tekst się nie zmieści. Font jest referencja do pliku zawierającego czcionkę którą renderowany jest tekst. Alignment odpowiada za to wyrównanie tekstu. W tym przypadku tekst jest wyśrodkowany zarówno w poziomie jak i w pionie. Ponadto panel ten zawiera jeszcze guzik który pozwala na zamknięcie okna. Jest on realizowany przez obiekt zawierający komponenty typu Button oraz Image. Komponent Button reaguje na eventy związane ze wskaźnikiem myszy. Wartość translation czyli przejście jest ustawione na Color Tint - zmiana koloru. Zależenie od tego czy przycisk jest w stanie domyślnym, znajduje się na nim myszka, jest kliknięty lub wyłączony zmienia się kolor obrazka ustawionego w komponencie Image. Pole On Click () odpowiada za przypisanie akcji które dzieją się po kliknięciu guzika. W tym przypadku podpięta jest jedna z funkcji skryptu MainMenuUI z głównego obiektu "MainMenuUI".

\begin{figure}[H]
	\begin{center}
\centering
\includegraphics[scale=0.7]{\ImgPath/rys/text_component.png}
\end{center}
	\caption{Komponent Text}
	\label{text_component}
\end{figure}


\begin{figure}[H]
	\begin{center}
\centering
\includegraphics[scale=0.7]{\ImgPath/rys/button_component.png}
\end{center}
	\caption{Komponent Button}
	\label{button_component}
\end{figure}

Panel ButtonsPanel zawiera zestaw guzików do nawigacji w grze. Wszystkie z nich są niemalże identyczne. Różnią się tylko pozycja, tekstem oraz podpiętymi funkcjami. Sam panel zawiera dodatkowo komponent vertical Layout Group. Jest on odpoweidzialny za automatyczne równie ustawienie elementów UI w pionie. Komponent jest tak ustawiony że elementy wyrównane są do środka panelu. Pierwszy Guzik - Local Match - służy do rozegrania meczu na 2 graczy na komputerze lokalnym. Guzik Online Match przenosi nas do oddzielnego menu w którym znajdujemy partnera do gry meczu 2 osobowego online. Przycisk Credits otwiera wcześniej opisany CreditsPanel. Ostatni z przycisków - Exit, zamyka grę. Wszystkie guziki wykorzystują metody znajdujące się w skrypcie MainMenuUI.

Jednym z ważniejszych elementów sceny jest obiekt GameState. Zawiera on dane które muszą być dostępne w każdym skrypcie, dlatego też skrypt przypięty do niego zrealizowany jest jako singleton. Singleton w Unity realizowany jest poprzez stworzenie statycznej referencji do konkretnego obiektu. Obiekt takiej klasy podczas tworzenia sprawdza czy globalna instancja istnieje i jeżeli nie referencja zaczyna wskazywać na niego. W przeciwnym wypadku nowy obiekt jest od razu niszczony. Sam skrypt GameState jest głównie konternerem na dane i ma bardzo mało funkcji. Dane które przechowuje ten skrypt to:
\begin{itemize}
    \item isMultiplayer - prosta flaga mówiąca  o tym czy aplikacja działa w trybie online czy nie. Wiele funkcji w logice rozgrywki często sprawdza tą flagę żeby wiedzieć jak realizować swoje zdania.
    \item defaultGameData - referencja do instancji Scriptable Objectu typu GameDefaultData. Przechowuje on wartości domyślne ustawień meczu, dostępne w grze warianty kolorystyczne piłki i graczy, a także listę map. Obiekty typu Scriptable Object w odróżnieniu do Mono Behaviourów nie są instancjonowanie na scenie gry a w plikach na dysku. Instancja GameDefaultData znajduje się w projekcie gry a jej zawartość można zobaczyć na rysunku  \ref{default_game_data_instance}. Obiekt ten służy wyłącznie do odczytu i nie jest modyfikowany podczas gry.
    \item currentMatchSetup - obecne ustawienia meczu. Podczas inicjalizacji obiektu są one kopiowane z defaultGameData. Potem podczas gry gracz może je zmieniać poprzez okno ustawień meczu. Zawierają informacje o długości meczu, wybranej mapie, wariancie kolorystycznym piłki oraz graczy, ilości graczy oraz czasie odliczania przed meczem. Ostatnia wartość nie jest dostępna dla graczy, służy głównie do synchronizacji czasu w meczu online.
\end{itemize}

\begin{figure}[H]
	\begin{center}
\centering
\includegraphics[width = \textwidth]{\ImgPath/rys/default_game_data_instance.png}
\end{center}
	\caption{Instancja GameDefaultData w grze}
	\label{default_game_data_instance}
\end{figure}

Na scenie znajduje się również obiekt MatchSetupUI. Odpowiada on za okno ustawień meczu - rysunek \ref{match_setup}. Obiekt ten  jest domyślnie ukryty i pojawia się dopiero gdy użytkownik będzie chciał uruchomić mecz. Okno używane jest zarówno w trybie lokalnym jak i online dlatego właśnie obiekt jest prefabem. Znajduje się również na scenie MultiplayerJoin. Z tych powodów obiekt posiada też oddzielne canvas - bez względu gdzie się znajduje, będzie działać niezależnie od reszty obiektów na scenie. W składzie okna znajdują się:
\begin{itemize}
    \item Wybór czasu meczu,
    \item Wybór wariantu piłki,
    \item Wybór mapy
    \item Wybór wariantów kolorów postaci graczy,
    \item Guziki: CANCEL oraz READY.
\end{itemize}
Każdy z elementów sterujących wyborem zawiera strzałki do zmiany aktualnie wybranej pozycji. Kliknięcie przycisku READY oznacza że akceptujemy ustawienia meczu i chcemy zacząć już rozgrywkę. Guzik CANCEL wyłącza okno i pozwala wrócić do głównego menu.

\begin{figure}[H]
	\begin{center}
\centering
\includegraphics[width = \textwidth]{\ImgPath/rys/match_setup.png}
\end{center}
	\caption{Okno ustawień meczu}
	\label{match_setup}
\end{figure}

Kolejną sceną w grze jest scena o nazwie "MutiplayerJoin". Znajduje się w niej głównie UI oraz inne obiekty związane z łączeniem się przez sieć z innym graczem. Scena jest wczytywana po wybraniu guzika Online Match w menu głównym. Funkcjonalność zostanie sceny zostanie szczegółowiej opisana w podrozdziale \ref{implementation_online}.

Ostatnie 2 sceny w grze są dość podobne - obydwie służą do samej rozgrywki. Różnią się głównie wyglądem tła oraz koszów. Mają też nieco zmienione Collidery. Na potrzeby tej pracy opisana zostanie tylko pierwsza z nich - COURT\_1




% Coś o spawnerze piłek

% Ustawinia kamery


\section{Implementacja trybu online multiplayer}
\label{implementation_online}
Meeh się robiło

\section{Przygotowanie gry pod nowe platformy}

całkiem ok

%-----------------
% Testowanie
%-----------------
\chapter{Przetestowanie opracowanej gry komputerowe}


Ważne, pameitaj że w poprzednim roku na techniki multimedialne robiłeś samasymulacje piłki - były to ponieka testy.

\section{Gameplay}

\section{Wydajność}

%-----------------
% Podsumowanie i wnioski 
%-----------------
\chapter{Podsumowanie i wnioski }

\begin{thebibliography}{99}
\addcontentsline{toc}{chapter}{Bibliografia}
\bibitem{gamejolt_page}{Gra Turbo Slam Dunk Unleashed w serwise GameJolt. \url{https://gamejolt.com/games/turbo-slam-dunk-unleashed/123485} [dostęp 14 stycznia 2019]}
\bibitem{unity_wiki} {Strona Wikipedi poświęcona Unity  \url{https://en.wikipedia.org/wiki/Unity_(game_engine)} [dostęp 17 stycznia 2019]}
\bibitem{canalplus_gry} {Twórcy Światów - Dokument zrealizowany przez Canal + \url{https://www.canalplus.pl/discovery/tworcy-swiatow}}
\bibitem{wiki_gameshistory} {Historia gier komputerowych na Wikipedi: \url{https://pl.wikipedia.org/wiki/Historia_gier_komputerowych} [dostęp 19 stycznia 2019]}
\bibitem{about_mono} {Oficjalna strona Mono \url{https://www.mono-project.com/docs/about-mono/} [dostęp 20 stycznia 2019]}

\end{thebibliography}

\listoffigures
 
\listoftables

\appendix
\chapter{Załączniki}
    \definecolor{codegreen}{rgb}{0,0.5,0}
\definecolor{codegray}{rgb}{0.5,0.5,0.5}
\definecolor{codepurple}{rgb}{0.58,0,0.82}
\definecolor{backcolour}{rgb}{0.95,0.95,0.97}
\definecolor{bluekeywords}{rgb}{0,0,1}
\definecolor{redstrings}{rgb}{0.64,0.08,0.08}
 
\lstdefinestyle{mystyle}{
    morekeywords={partial, var, value, get, set, using, class, public},
    backgroundcolor=\color{backcolour},   
    commentstyle=\color{codegreen},
    keywordstyle=\color{bluekeywords},
    numberstyle=\tiny\color{codegray},
    stringstyle=\color{redstrings},
    breakatwhitespace=false,         
    breaklines=true,                 
    captionpos=b,                    
    keepspaces=true,                 
    numbers=left,                    
    numbersep=5pt,                  
    showspaces=false,                
    showstringspaces=false,
    showtabs=false,                  
    tabsize=2,
    language=csh,
    basicstyle=\footnotesize\ttfamily,
    numbersep=5pt,
    extendedchars=true,
    frame=b,
    xleftmargin=17pt,
    framexleftmargin=17pt,
    framexrightmargin=5pt,
    framexbottommargin=4pt,
    commentstyle=\color{green},
    morecomment=[l]{//}, %use comment-line-style!
    morecomment=[s]{/*}{*/}, %for multiline comments
}
 
\lstset{style=mystyle}

\section{Kod skryptu MainMenuUI.cs}
    \lstinputlisting[language={[Sharp]C}, caption={Kod skryptu MainMenuUI.cs}, label={code_MainMenuUI}]{\ImgPath/Code/MainMenuUI.txt}

\section{Klasa InputPlayer}
    \lstinputlisting[language={[Sharp]C}, caption={Klasa InputPlayer}, label={code_InputPLayer}]{\ImgPath/Code/InputPlayer.txt}

\section{Wybrane funkcje skryptu UniverseManager.cs}
    \lstinputlisting[language={[Sharp]C}, caption={Wybrane funkcje skryptu UniverseManager.cs}, label={code_UniverseManager}]{\ImgPath/Code/UniverseManager_functions.txt}
    
\section{Przykładowe zdarzenie sieciowe}
    \lstinputlisting[language={[Sharp]C}, caption={Przykład zdarzenia sieciowego odpowiedzialnego za Zwiększenie liczby fauli}, label={code_PhotonEventExample}]{\ImgPath/Code/PhotonEventExample.txt}

%\zakonczenie  % wklejenie recenzji i opinii

\end{document}
%+++ END +++
